\documentclass[12pt]{article}
\usepackage[brazil]{babel}
\usepackage[utf8]{inputenc}
\usepackage{amsmath}
\usepackage{natbib}
\usepackage{listings}
\usepackage{color}

\definecolor{codegreen}{rgb}{0,0.6,0}
\definecolor{codegray}{rgb}{0.5,0.5,0.5}
\definecolor{codepurple}{rgb}{0.58,0,0.82}
\definecolor{backcolour}{rgb}{0.95,0.95,0.92}

\lstdefinestyle{mystyle}{
    backgroundcolor=\color{backcolour},
    commentstyle=\color{codegreen},
    keywordstyle=\color{red},
    numberstyle=\tiny\color{codegray},
    stringstyle=\color{codepurple},
    basicstyle=\footnotesize,
    breakatwhitespace=false,
    breaklines=true,
    captionpos=b,
    keepspaces=true,
    numbers=left,
    numbersep=5pt,
    showspaces=false,
    showstringspaces=false,
    showtabs=false,
    tabsize=2
}

\lstset{style=mystyle}
\usepackage{url}
\usepackage{amsmath}
\usepackage{float}
\usepackage{graphicx}
\graphicspath{{images/}}
\usepackage{parskip}
\usepackage{fancyhdr}
\usepackage{vmargin}
\setmarginsrb{3 cm}{2.5 cm}{3 cm}{2.5 cm}{1 cm}{1.5 cm}{1 cm}{1.5 cm}

\title{Zeros de Funções }								% Title
\author{Wilton Rodrigues}								% Author
\date{\today}											% Date

\makeatletter
\let\thetitle\@title
\let\theauthor\@author
\let\thedate\@date
\makeatother

\pagestyle{fancy}
\fancyhf{}
\lhead{\centering{\thetitle}}
\cfoot{\thepage}

\begin{document}

%%%%%%%%%%%%%%%%%%%%%%%%%%%%%%%%%%%%%%%%%%%%%%%%%%%%%%%%%%%%%%%%%%%%%%%%%%%%%%%%%%%%%%%%%

\begin{titlepage}
  \centering
  \begin{figure}[H]
    \centering
    \includegraphics[width=0.7\textwidth]{logo.png}\\[2.0 cm]
  \end{figure}
  \textsc{\LARGE Universidade de Brasília}\\[2.5 cm]	% University Name
  \textsc{\Large Relatório de atividade do módulo 1}\\[0.5 cm]				% Activity name
  \textsc{\large Métodos Numéricos para Engenharia}\\[1.5 cm]				% Course Name
  \rule{\linewidth}{0.2 mm} \\[0.4 cm]
  {\huge \bfseries \thetitle}\\
  \rule{\linewidth}{0.2 mm} \\[2.5 cm]

  \begin{minipage}{0.4\textwidth}
    \begin{flushleft} \large
      \emph{Aluno:}\\
      \theauthor
    \end{flushleft}
  \end{minipage}
  \begin{minipage}{0.4\textwidth}
    \begin{flushright} \large
      \emph{Matrícula:} \\
      13/0049212									% Your Student Number
    \end{flushright}
  \end{minipage}\\
  \vspace*{0.5in}
  {\large \thedate}\\[0.5 cm]

  \vfill

\end{titlepage}

%%%%%%%%%%%%%%%%%%%%%%%%%%%%%%%%%%%%%%%%%%%%%%%%%%%%%%%%%%%%%%%%%%%%%%%%%%%%%%%%%%%%%%%%%
\tableofcontents
\pagebreak
%%%%%%%%%%%%%%%%%%%%%%%%%%%%%%%%%%%%%%%%%%%%%%%%%%%%%%%%%%%%%%%%%%%%%%%%%%%%%%%%%%%%%%%%%

\section{Introdução}

O objetivo deste relatório é exercitar os conceitos aprendidos em aula, com relação ao tema Zeros de funções. O objetivo deste tópico é prover métodos matemáticos capazes de determinar o ponto, ou pontos, nos quais a equação cruza ou toca o eixo X, ou seja, um valor numérico que satisfaça à equação. Tarefa que pode dispender um enorme esforço, ou em alguns casos é até mesmo impossível, em equações que não possuem solução analítica. Como é o caso da Equação de Kepler que é dada por:
\begin{equation} \label{eq:original}
  M = x - E sin(x)
\end{equation}
Dado que $E = 0.2$ e $M = 0.5$, o objetivo deste trabalho é obter a raíz da equação~\eqref{eq:original} com precisão de 10 casas decimais.

Para isto, utilizaremos dois passos: O passo 1 tem como objetivo obter um intervalo $[a,b]$ aproximado no qual $x \in [a,b]$. O passo 2 é onde aplicaremos o método numérico da Bissecção para refinar a solução.

\section{Diagrama esquemático de execução}

\newpage
\section{Código Fonte}

\lstinputlisting[language=C]{../solution/m1.c}

\newpage
\section{Resultados e discussões}

%%%%%%%%%%%%%%%%%%%%%%%%%%%%%%%%%%%%%%%%%%%%%%%%%%%%%%%%%%%%%%%%%%%%%%%%%%%%%%%%%%%%%%%%%

\end{document}
